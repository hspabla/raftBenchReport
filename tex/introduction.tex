\section{Introduction}
 
While designing a fault tolerant distributed system, consensus management becomes a key and fundamental problem. It is to agree on a single value between multiple servers under any failure model. Once agreed upon, the decision is intact and fixed. In this paper, our reference to failure model points to non-byzantine failures only. This need for consensus arises when we replicate state machines across servers to support fault-tolerance. While the client looks at this system as one server which responds to their requests, this consensus module ensures each server executes the same commands in the same order so as to produce the same result. Typically, this agreement process involves getting a majority of the servers to agree upon a value, which allows for continued operation even when a minority of the servers fail. One of the original solutions to consensus problem was Paxos [cite paxos] by Leslie Lamport. While it is highly popular and used in quite a few systems, it is considered to be a highly convoluted algorithm, which is difficult to understand, realized even further while implementing it as a real system. Furthermore, classical Paxos has many performance flaws, which were fixed in follow up papers such as Mencius\cite{mencius}, Egalitarian Paxos\cite{epaxos} etc. While there are research papers proposing easier solutions for Paxos, many lacked mathematically or experimentally proven good and complete implementable solutions. 

As a result, Raft was introduced to provide a simpler to understand, easy to implement solution for consensus building. The consensus problems, such as leader election, log replication, safety were decomposed into independent sub-problems for improving understandability, while making the algorithm more intuitive by reducing the degree of non-determinism and inconsistency amongst the servers. 
The original Raft research paper provided evaluations in three aspects: understandability , wherein students were made to take a Paxos and Raft quiz and their marks were compared; correctness, which provided a formal specification and a proof of safety; and performance, to measure its leader election operation. However, it did not evaluate Raft in terms of its execution time or failure tolerance under contention and under different workloads. 

With this paper, we answer the below research questions:
\begin{enumerate}
\item How does RAFT system perform under contention and without contention?
\item How does RAFT perform on the following workloads: read-only, write-only, equal read-write workload.
\end{enumerate}

The remainder of this paper is structured as follows: In Section 2, we gives a brief overview of the Raft consensus module. Section 3 covers the experimental setup and methodology used to measure performance. Section 4 details the results obtained from our experiments, their corresponding graphs and our insights into the design decisions made in Raft that could have caused the results. We provide potential future work in Section 5. Section 6 offers concluding remarks followed by acknowledgements.
